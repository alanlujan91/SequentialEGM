\input{./econtexRoot.texinput}
\documentclass[\econtexRoot/HAFiscal]{subfiles}
\onlyinsubfile{\externaldocument{\econtexRoot/HAFiscal}} % Get xrefs -- esp to apndx -- from main file; only works if main file has already been compiled

\begin{document}

\hypertarget{introduction}{}\par\section{Introduction}\notinsubfile{\label{sec:intro}}
\setcounter{page}{0}\pagenumbering{arabic}

\cite{Ludwig2018-uz,Barillas2007-uh,Carroll2005-kf,Carroll2006-wq,Druedahl2017-vn,Druedahl2017-lt,Druedahl2021-wl,Fella2011-dc,Fella2014-my,Hintermaier2010-io,Ishakov2012-jm,Iskhakov2012-rt,Iskhakov2015-xw,Iskhakov2015-jy,Iskhakov2016-xb,Iskhakov2017-my,Jang2018-lz,Jang2019-da,Jang2021-wu,Jorgensen2013-du,Judd2010-of,Ludwig2016-tq,Maliar2011-dj,Maliar2013-sv,Maliar2013-ys,Maliar2014-qj,Mendoza2020-jd,Rust2012-yx,Rust2012-vm,Schon_undated-jf,Villemot2012-tk,Villemot2012-np,White2014-ng,White2015-fg,Iskhakov2017-mz,Iskhakov2017-pf,Iskhakov2016-if,Fella2011-ro,Clausen2020-zo,Maliar2013-kl,Iskhakov2017-uf,Iskhakov2016-wb,Maliar2013-lo,Fella2011-tw,Iskhakov2017-do}

\subsection{Background} % Introduce the topic by providing a brief overview of the issue and why it is important to study it.

% Identify the research question: Clearly state the research question or problem being addressed in the current study.
% Provide context: Explain why the topic is important to study and what gap in the existing knowledge the current study aims to fill.
% Summarize the existing literature: Briefly describe what is currently known about the topic, including any relevant studies or theories that have been previously published.
% Highlight the limitations of previous research: Identify any limitations or gaps in the existing literature and explain how the current study will address these limitations.
% Provide a rationale for the study: Explain why the current study is needed and why it is a significant contribution to the existing knowledge on the topic.

% Use only the first paragraph to state the question and describe its importance. Don't weave
% around, be overly broad, or use prior literature to motivate it (the question is not important
% because so many papers looked at this issue before!).

% Then use the second paragraph for a summary of the most relevant literature
% (not a full section!). Hint: use present tense, to be consistent. "Smith (1986) presents a similar model, ..."

% Next, while still on page one, the third paragraph must begin: "The purpose of this paper is ...",
% and summarize what you actually do. (Paragraphs 2 and 3 could be reversed.)

% That sets you up for the fourth paragraph, which lists "The contributions of
% this work" – relative to that prior literature. Clarify what you do that's different

% The fifth paragraph then summarizes your results. Tell the answer, so they know what to expect,
% and how to think about each step along the way, what's driving your results.

% In the sixth and final paragraph, as an aid to the reader, plot the course for the rest of the paper.
% "The first section below presentsa theoretical model that can be used to generate specific
% hypotheses. Then section 2 presen ts the econometric model, ..."

The endogenous grid method (EGM) developed by (Carroll, 1997) has allowed the solving of dynamic optimization problems to be done in a computationally efficient and fast manner. Many problems that before took hours to solve became much more easier to solve and allowed us to focus on estimation and simulation.
However, the engodenous grid method is limited to a few class of problems. Recently, the class of problems to which EGM can be applied has been expanded by (cite a few papers), but with every new method comes a new set of limitations.
This paper introduces a new approach to EGM in a multivariate setting. The method is called Sequential EGM (or EGM$^N$) and introduces a novel way of breaking down complex problems into a sequence of simpler, smaller, and more tractable problems, along with an exploration of new multidimensional interpolation methods that can be used to solve these problems.

\subsection{Literature Review} % Summarize the existing literature on the topic and highlight any gaps or limitations in the current research.

The literature review should cite the following:



\cite{Druedahl2021-wl,Ludwig2018-uz,Ludwig2016-tq,Iskhakov2015-jy,Maliar2013-sv,Carroll2006-wq,Jorgensen2013-du,Maliar2011-dj,White2015-fg,Hintermaier2010-io,Barillas2007-uh,Druedahl2017-vn,Iskhakov2017-my,Mendoza2020-jd,Fella2014-my}




\subsection{Research Question} % Clearly state the research question or problem being addressed in the current study.

\subsection{Methodology} % Briefly describe the research methodology used in the study, including any data sources, econometric techniques, or other methods used.

\subsection{Contributions} %  Discuss how the current study contributes to the existing literature and what new insights it provides.

\subsection{Outline} %  Provide a brief overview of the results and conclusions that will be presented in the article.

\onlyinsubfile{\input{\LaTeXInputs/bibliography_blend}}

\ifthenelse{\boolean{Web}}{}{
\onlyinsubfile{\captionsetup[figure]{list=no}}
\onlyinsubfile{\captionsetup[table]{list=no}}
\end{document} \endinput
}

